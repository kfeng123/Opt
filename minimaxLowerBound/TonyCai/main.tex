\documentclass[11pt, letterpaper]{article}
 
\usepackage{lineno,hyperref}

\usepackage{galois} % composition function \comp
\usepackage{bm}
\usepackage{amsmath}
\usepackage{amssymb}
\usepackage{mathrsfs}
\usepackage{amsthm}
\usepackage{graphicx}
\usepackage{color}
\usepackage{booktabs}
\usepackage{datetime}
\newdate{date}{9}{1}{2017}

%%%%%%%%%%%%%%  Notations %%%%%%%%%%
\DeclareMathOperator{\mytr}{tr}
\DeclareMathOperator{\mydiag}{diag}
\DeclareMathOperator{\myrank}{Rank}
\DeclareMathOperator{\myP}{P}
\DeclareMathOperator{\myE}{E}
\DeclareMathOperator{\myVar}{Var}
\DeclareMathOperator*{\argmax}{arg\,max}
\DeclareMathOperator*{\argmin}{arg\,min}


\newcommand{\Ba}{\mathbf{a}}    \newcommand{\Bb}{\mathbf{b}}    \newcommand{\Bc}{\mathbf{c}}    \newcommand{\Bd}{\mathbf{d}}    \newcommand{\Be}{\mathbf{e}}    \newcommand{\Bf}{\mathbf{f}}    \newcommand{\Bg}{\mathbf{g}}    \newcommand{\Bh}{\mathbf{h}}    \newcommand{\Bi}{\mathbf{i}}    \newcommand{\Bj}{\mathbf{j}}    \newcommand{\Bk}{\mathbf{k}}    \newcommand{\Bl}{\mathbf{l}}
\newcommand{\Bm}{\mathbf{m}}    \newcommand{\Bn}{\mathbf{n}}    \newcommand{\Bo}{\mathbf{o}}    \newcommand{\Bp}{\mathbf{p}}    \newcommand{\Bq}{\mathbf{q}}    \newcommand{\Br}{\mathbf{r}}    \newcommand{\Bs}{\mathbf{s}}    \newcommand{\Bt}{\mathbf{t}}    \newcommand{\Bu}{\mathbf{u}}    \newcommand{\Bv}{\mathbf{v}}    \newcommand{\Bw}{\mathbf{w}}    \newcommand{\Bx}{\mathbf{x}}
\newcommand{\By}{\mathbf{y}}    \newcommand{\Bz}{\mathbf{z}}    
\newcommand{\BA}{\mathbf{A}}    \newcommand{\BB}{\mathbf{B}}    \newcommand{\BC}{\mathbf{C}}    \newcommand{\BD}{\mathbf{D}}    \newcommand{\BE}{\mathbf{E}}    \newcommand{\BF}{\mathbf{F}}    \newcommand{\BG}{\mathbf{G}}    \newcommand{\BH}{\mathbf{H}}    \newcommand{\BI}{\mathbf{I}}    \newcommand{\BJ}{\mathbf{J}}    \newcommand{\BK}{\mathbf{K}}    \newcommand{\BL}{\mathbf{L}}
\newcommand{\BM}{\mathbf{M}}    \newcommand{\BN}{\mathbf{N}}    \newcommand{\BO}{\mathbf{O}}    \newcommand{\BP}{\mathbf{P}}    \newcommand{\BQ}{\mathbf{Q}}    \newcommand{\BR}{\mathbf{R}}    \newcommand{\BS}{\mathbf{S}}    \newcommand{\BT}{\mathbf{T}}    \newcommand{\BU}{\mathbf{U}}    \newcommand{\BV}{\mathbf{V}}    \newcommand{\BW}{\mathbf{W}}    \newcommand{\BX}{\mathbf{X}}
\newcommand{\BY}{\mathbf{Y}}    \newcommand{\BZ}{\mathbf{Z}}    

\newcommand{\bfsym}[1]{\ensuremath{\boldsymbol{#1}}}

 \def\balpha{\bfsym \alpha}
 \def\bbeta{\bfsym \beta}
 \def\bgamma{\bfsym \gamma}             \def\bGamma{\bfsym \Gamma}
 \def\bdelta{\bfsym {\delta}}           \def\bDelta {\bfsym {\Delta}}
 \def\bfeta{\bfsym {\eta}}              \def\bfEta {\bfsym {\Eta}}
 \def\bmu{\bfsym {\mu}}                 \def\bMu {\bfsym {\Mu}}
 \def\bnu{\bfsym {\nu}}
 \def\btheta{\bfsym {\theta}}           \def\bTheta {\bfsym {\Theta}}
 \def\beps{\bfsym \varepsilon}          \def\bepsilon{\bfsym \varepsilon}
 \def\bsigma{\bfsym \sigma}             \def\bSigma{\bfsym \Sigma}
 \def\blambda {\bfsym {\lambda}}        \def\bLambda {\bfsym {\Lambda}}
 \def\bomega {\bfsym {\omega}}          \def\bOmega {\bfsym {\Omega}}
 \def\brho   {\bfsym {\rho}}
 \def\btau{\bfsym {\tau}}
 \def\bxi{\bfsym {\xi}}
 \def\bzeta{\bfsym {\zeta}}
% May add more in future.
%%%%%%%%%%%%%%%%%%%%%%%%%%%%%%%%%%%%



\theoremstyle{plain}
\newtheorem{theorem}{\quad\quad Theorem}
\newtheorem{proposition}{\quad\quad Proposition}
\newtheorem{corollary}{\quad\quad Corollary}
\newtheorem{lemma}{\quad\quad Lemma}
\newtheorem{example}{Example}
\newtheorem{assumption}{\quad\quad Assumption}
\newtheorem{condition}{\quad\quad Condition}

\theoremstyle{definition}
\newtheorem{remark}{\quad\quad Remark}
\theoremstyle{remark}


\begin{document}
\title{haha}
\maketitle
\section{Introduction}
\section{bounds for the radius of confidence balls}
These results are from Cai and Low 2004.
I adapted it from ``All of Nonparametric Statistics''.

Let $\BZ^n=(Z_1,\ldots,Z_n)$ where $Z_{i}=\theta_i+\sigma_n\epsilon_n$, $i=1,\ldots,n$, $\epsilon_1,\ldots,\epsilon_n$ are independent $N(0,1)$ random variables, $\theta^n=(\theta_1,\ldots,\theta_n)\in\mathbb{R}^n$ is a vector of unknown parameters and $\sigma_n$ is assumed known. 
\begin{theorem}[Cai and Low 2004]
    Fix $0<\alpha<1/2$. Let $\mathcal{B}_n=\{\theta:\|\hat{\theta}-\theta\|\leq s_n\}$ be such that
    $$
    \inf_{\theta\in \mathbb{R}^n} \myP_{\theta}(\theta\in \mathcal{B}_n)\geq 1-\alpha.
    $$
    Then, for every $0<\epsilon<1/2-\alpha$,
    $$
    \inf_{\theta\in \mathbb{R}^n}\myE_{\theta}(s_n)\geq \frac{1}{2}\sigma_n(1-2\alpha-\epsilon)n^{1/4}(\log(1+\epsilon^2))^{1/4}.
    $$
\end{theorem}
\begin{proof}
    Let 
    $$
    a=\frac{\sigma_n}{n^{1/4}}(\log(1+\epsilon^2))^{1/4}
    $$
    and define
    $$
    \Omega=\{\theta=(\theta_1,\ldots,\theta_n):|\theta_i|=a,\,i=1,\ldots,n\}.
    $$
    Note that $\Omega$ contains $2^n$ elements. Let $f_{\theta}$ denote the density of a multivariate normal with mean $\theta$ and covariance $\sigma_n^2 I$ where $I$ is the identity matrix.
    Define the mixture
    $$
    q(y)=\frac{1}{2^n}\sum_{\theta\in\Omega}f_{\theta}(y).
    $$
    Let $f_0$ denote the density of a multivariate normal with mean $(0,\ldots,0)$ and covariance $\sigma_n^2 I$.

    It can be proved that $\int |f_0(x)-q(x)|dx\leq \epsilon$.
    
    Define two events, $A=\{(0,\ldots,0)\in \mathcal{B}_n\}$ and $B=\{\Omega\cap \mathcal{B}_n \neq \emptyset\}$.
    Every $\theta\in \Omega$ has norm
    $$
    \|\theta\|=\sqrt{n a^2}\overset{def}{=}c_n.
    $$
    Hence, $A\cap B\subset\{2s_n\geq c_n\}$.
    For all $\theta\in\Omega$, we have
    $$
    \myP_\theta(\Omega\cap\mathcal{B}_n\neq \emptyset)\geq
    \myP_\theta(\theta\in\mathcal{B}_n)\geq 1-\alpha.
    $$
    Hence, $Q(B)\geq 1-\alpha$ and thus $\myP_0(B)\geq 1-\alpha-\epsilon$. Then
    $$
    \myP_0(2s_n\geq c_n)\geq 
    \myP_0(A\cap B)\geq \myP_0(A)+\myP_0(B)-1
    \geq 1-2\alpha-\epsilon
    $$

\end{proof}
\begin{theorem}
    Fix $0<\alpha<1/2$. Let $\mathcal{B}_n=\{\theta:\|\hat{\theta}-\theta\|\leq s_n\}$ be such that
    $$
    \inf_{\theta\in \mathbb{R}^n} \myP_{\theta}(\theta\in \mathcal{B}_n)\geq 1-\alpha.
    $$
    Then, for every $0<\epsilon<1/2-\alpha$,
    $$
    \sup_{\theta\in \mathbb{R}^n} \myE_{\theta}(s_n)\geq \epsilon \sigma_n z_{\alpha+2\epsilon}\sqrt{n}\sqrt{\frac{\epsilon}{1-\alpha-\epsilon}}.
    $$

\end{theorem}
\section*{Acknowledgements}


\section*{References}

\bibliographystyle{apalike}
\bibliography{mybibfile}

\end{document}
