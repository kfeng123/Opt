\documentclass[11pt]{article}
 
\newcommand\CG[1]{\textcolor{red}{#1}}

\usepackage{lineno,hyperref}

\usepackage[margin=1 in]{geometry}
\renewcommand{\baselinestretch}{1.25}


%\usepackage{refcheck}
\usepackage{authblk}
\usepackage{galois} % composition function \comp
\usepackage{bm}
\usepackage{amsmath}
\usepackage{amssymb}
\usepackage{mathrsfs}
\usepackage{amsthm}
\usepackage{natbib}
\usepackage{graphicx}
\usepackage{color}
\usepackage{booktabs}
\usepackage[page,title]{appendix}
%\renewcommand\appendixname{haha}
\usepackage{enumerate}
\usepackage{changepage}
\usepackage{datetime}
\newdate{date}{9}{1}{2017}

%%%%%%%%%%%%%%  Notations %%%%%%%%%%
\DeclareMathOperator{\mytr}{tr}
\DeclareMathOperator{\mydiag}{diag}
\DeclareMathOperator{\myrank}{Rank}
\DeclareMathOperator{\myP}{P}
\DeclareMathOperator{\myE}{E}
\DeclareMathOperator{\myVar}{Var}
\DeclareMathOperator{\myCov}{Cov}
\DeclareMathOperator*{\argmax}{arg\,max}
\DeclareMathOperator*{\argmin}{arg\,min}


\newcommand{\Ba}{\mathbf{a}}    \newcommand{\Bb}{\mathbf{b}}    \newcommand{\Bc}{\mathbf{c}}    \newcommand{\Bd}{\mathbf{d}}    \newcommand{\Be}{\mathbf{e}}    \newcommand{\Bf}{\mathbf{f}}    \newcommand{\Bg}{\mathbf{g}}    \newcommand{\Bh}{\mathbf{h}}    \newcommand{\Bi}{\mathbf{i}}    \newcommand{\Bj}{\mathbf{j}}    \newcommand{\Bk}{\mathbf{k}}    \newcommand{\Bl}{\mathbf{l}}
\newcommand{\Bm}{\mathbf{m}}    \newcommand{\Bn}{\mathbf{n}}    \newcommand{\Bo}{\mathbf{o}}    \newcommand{\Bp}{\mathbf{p}}    \newcommand{\Bq}{\mathbf{q}}    \newcommand{\Br}{\mathbf{r}}    \newcommand{\Bs}{\mathbf{s}}    \newcommand{\Bt}{\mathbf{t}}    \newcommand{\Bu}{\mathbf{u}}    \newcommand{\Bv}{\mathbf{v}}    \newcommand{\Bw}{\mathbf{w}}    \newcommand{\Bx}{\mathbf{x}}
\newcommand{\By}{\mathbf{y}}    \newcommand{\Bz}{\mathbf{z}}    
\newcommand{\BA}{\mathbf{A}}    \newcommand{\BB}{\mathbf{B}}    \newcommand{\BC}{\mathbf{C}}    \newcommand{\BD}{\mathbf{D}}    \newcommand{\BE}{\mathbf{E}}    \newcommand{\BF}{\mathbf{F}}    \newcommand{\BG}{\mathbf{G}}    \newcommand{\BH}{\mathbf{H}}    \newcommand{\BI}{\mathbf{I}}    \newcommand{\BJ}{\mathbf{J}}    \newcommand{\BK}{\mathbf{K}}    \newcommand{\BL}{\mathbf{L}}
\newcommand{\BM}{\mathbf{M}}    \newcommand{\BN}{\mathbf{N}}    \newcommand{\BO}{\mathbf{O}}    \newcommand{\BP}{\mathbf{P}}    \newcommand{\BQ}{\mathbf{Q}}    \newcommand{\BR}{\mathbf{R}}    \newcommand{\BS}{\mathbf{S}}    \newcommand{\BT}{\mathbf{T}}    \newcommand{\BU}{\mathbf{U}}    \newcommand{\BV}{\mathbf{V}}    \newcommand{\BW}{\mathbf{W}}    \newcommand{\BX}{\mathbf{X}}
\newcommand{\BY}{\mathbf{Y}}    \newcommand{\BZ}{\mathbf{Z}}    

\newcommand{\bfsym}[1]{\ensuremath{\boldsymbol{#1}}}

 \def\balpha{\bfsym \alpha}
 \def\bbeta{\bfsym \beta}
 \def\bgamma{\bfsym \gamma}             \def\bGamma{\bfsym \Gamma}
 \def\bdelta{\bfsym {\delta}}           \def\bDelta {\bfsym {\Delta}}
 \def\bfeta{\bfsym {\eta}}              \def\bfEta {\bfsym {\Eta}}
 \def\bmu{\bfsym {\mu}}                 \def\bMu {\bfsym {\Mu}}
 \def\bnu{\bfsym {\nu}}
 \def\btheta{\bfsym {\theta}}           \def\bTheta {\bfsym {\Theta}}
 \def\beps{\bfsym \varepsilon}          \def\bepsilon{\bfsym \varepsilon}
 \def\bsigma{\bfsym \sigma}             \def\bSigma{\bfsym \Sigma}
 \def\blambda {\bfsym {\lambda}}        \def\bLambda {\bfsym {\Lambda}}
 \def\bomega {\bfsym {\omega}}          \def\bOmega {\bfsym {\Omega}}
 \def\brho   {\bfsym {\rho}}
 \def\btau{\bfsym {\tau}}
 \def\bxi{\bfsym {\xi}}
 \def\bzeta{\bfsym {\zeta}}
% May add more in future.
%%%%%%%%%%%%%%%%%%%%%%%%%%%%%%%%%%%%



\theoremstyle{plain}
\newtheorem{theorem}{\quad\quad Theorem}
\newtheorem{proposition}{\quad\quad Proposition}
\newtheorem{corollary}{\quad\quad Corollary}
\newtheorem{lemma}{\quad\quad Lemma}
\newtheorem{example}{Example}
\newtheorem{assumption}{\quad\quad Assumption}
\newtheorem{condition}{\quad\quad Condition}

\theoremstyle{definition}
\newtheorem{definition}{\quad\quad Definition}
\newtheorem{remark}{\quad\quad Remark}
\theoremstyle{remark}



\title{
    Stein method for quadratic forms
}



\author[1]{Rui Wang}
%\author[2]{xx}
%\author[1,3]{Xingzhong Xu\thanks{Corresponding author\\Email address: xuxz@bit.edu.cn}}
%\affil[1]{
%School of Mathematics and Statistics, Beijing Institute of Technology, Beijing 
    %100081,China
%}
%\affil[2]{
    %xx
%}
%\affil[3]{
%Beijing Key Laboratory on MCAACI, Beijing Institute of Technology, Beijing 100081,China
%}



\begin{document}
\maketitle
    \section{haha2}
\begin{theorem}
    Let $\zeta_1,\ldots,\zeta_d$ be iid random variables with mean $0$ and variance $1$, and assume $\mu_k:= \myE (\zeta_1^k)$ is finite for $k\leq 8$.
    Let $\bzeta=(\zeta_1,\ldots,\zeta_d)^\top\in \mathbb R^d$.
    For $k=1,\ldots, K$, let $\BQ_k=(q_{i_j}^{(k)})$ be a $d\times d$ symmetric matrix and let $\check\BQ_k=\mydiag(q_{11}^{(k)},\ldots, q_{dd}^{(k)})$, $\hat\BQ_k=\BI_d-\check\BQ_k$.
    Define $\hat{w}_k= \bzeta^\top \hat\BQ_k \bzeta$, $\check w_k = \bzeta^\top \check \BQ_k \bzeta -\mytr(\BQ_k)$, and
\begin{equation*}
W=
    \begin{pmatrix}
   \hat w_1\\
   \check w_1\\
   \vdots\\
   \hat w_K\\
   \check w_K
    \end{pmatrix}
    =
    \begin{pmatrix}
        \bzeta^\top \hat \BQ_1 \bzeta
        \\
        \bzeta^\top \check \BQ_1 \bzeta -\mytr (\BQ_1)
        \\
        \vdots\\
        \bzeta^\top \hat \BQ_1 \bzeta
        \\
        \bzeta^\top \check \BQ_1 \bzeta -\mytr (\BQ_1)
        \\
    \end{pmatrix}
    \in \mathbb R^{2K}.
\end{equation*}
Finally, let $Z\sim \mathcal N_{2K} (0,\BI_{2K})$ and $\BV = \myCov (W)$. 
There is an absolute constant $0<C<\infty$ such that
\begin{equation*}
    haha
\end{equation*}
\end{theorem}
\begin{proof}
    Let $f:\mathbb R^{2K} \to \mathbb R$ be a four-times differentiable function.
    From xxx, there is a $4-times$ differentiable function $g : \mathbb R^{2K}\to \mathbb R$ satisfying the Stein identity
\begin{equation*}
    \myE [f(W)]    - \myE [f(\BV^{1/2} W)]=
    \myE [\nabla^\top  \BV \nabla g(W) - W^\top \nabla g(W)]
\end{equation*}
and 
\begin{equation*}
    \left| \frac{\partial^k g(\Bx)}{\prod_{j=1}^k \partial x_{i_j}} \right| 
    \leq
    \frac{1}{k}
    \left| \frac{\partial^k f(\Bx)}{\prod_{j=1}^k \partial x_{i_j}} \right|
    \quad
    \text{for all $\Bx=(x_1,\ldots, x_{2K})^\top \in \mathbb R^{2K}$,
    $k=1,2,3$, and $i_j \in \{1,\ldots, 2K\}$.}
\end{equation*}
To prove the theorem, we bound
\begin{equation*}
    S=\myE [\nabla^\top \BV \nabla g(W) -W^\top \nabla g(W)].
\end{equation*}

Next, we use exchangeability.
Let $\bzeta' = (\zeta_1',\ldots, \zeta_d')^\top$ be an independent copy of $\zeta$, and let $\underline i \in \{1,\ldots, d\}$ be an independent and uniformly distributed random index.
Define the vector $W' \in \mathbb R^{2K}$ exactly as we defined $W$, except that $\bzeta_{\underline i}$ is replaced with $\bzeta_{\underline i}'$ throughout.
More precisely, let $e_i \in \mathbb R^d$ be the $i$th standard basis vector in $\mathcal R^d$ and define
\begin{equation*}
\begin{split}
    \hat w_k'=& (\bzeta + (\zeta_{\underline i}' - \zeta_{\underline i}) e_{\underline i})^\top 
    \hat \BQ_k
    (\bzeta + (\zeta_{\underline i}' - \zeta_{\underline i}) e_{\underline i})
\\
=& \hat w_k + 2 (\zeta_{\underline i}' - \zeta_{\underline i}) e_{\underline i}^\top \hat \BQ_k \bzeta,
\end{split}
\end{equation*}
\begin{equation*}
\begin{split}
    \check w_k'=& (\bzeta + (\zeta_{\underline i}' - \zeta_{\underline i}) e_{\underline i})^\top 
    \check \BQ_k
    (\bzeta + (\zeta_{\underline i}' - \zeta_{\underline i}) e_{\underline i})
    -\mytr(\BQ_k)
\\
=& \check w_k +  e_{\underline i}^\top \check \BQ_k e_{\underline i} ((\zeta_{\underline i}')^2 - \zeta_{\underline i}^2) ,
\end{split}
\end{equation*}
for $k=1,\ldots, K$.
Then $W'= (\hat w_1',\check w_1', \ldots, \hat w_K', \check w_K')^\top \in \mathbb R^{2K}$.
Its straightforward to verify that
\begin{equation*}
    \myE (\hat w_k' - \hat w_k|\bzeta)=-\frac{2}{d} \hat w_k,
    \quad
    \myE (\check w_k' - \check w_k|\bzeta)=-\frac{1}{d} \check w_k.
\end{equation*}
Then
\begin{equation*}
    \myE ( W' - W |\bzeta) = - \Lambda_K W,
\end{equation*}
where
\begin{equation*}
    \Lambda_1=
    \begin{pmatrix}
       \frac 2 d & 0\\
       0 & \frac 1 d
    \end{pmatrix}
    ,
    \quad
    \Lambda_K =
    \begin{pmatrix}
        \Lambda_1 & 0 & \cdots & 0\\
         0 &\Lambda_1  & \cdots & 0\\
         \vdots & \vdots && \vdots\\
         0 & 0 & \cdots & \Lambda_1
    \end{pmatrix}
    \in \mathbb R^{2K\times 2K}.
\end{equation*}

By exchangeability, we have
\begin{equation*}
    \begin{split}
        0=&\frac 1 2 \myE [(W'-W)^\top \Lambda_K^{-\top} (\nabla g(W') +\nabla g(W))]
        \\
        =& \myE [(W'-W)^\top \Lambda_K^{-\top} \nabla g(W)]
        +\frac{1}{2} \myE [(W'-W)^\top \Lambda_K^{-\top} (\nabla g(W') -\nabla g(W))]
        \\
        =&
        - \myE [W^\top  \nabla g(W)]
        +\frac{1}{2} \myE [(W'-W)^\top \Lambda_K^{-\top} (\nabla g(W') -\nabla g(W))].
    \end{split}
\end{equation*}
That is,
\begin{equation*}
         \myE [W^\top  \nabla g(W)]
         =
        \frac{1}{2} \myE [(W'-W)^\top \Lambda_K^{-\top} (\nabla g(W') -\nabla g(W))].
\end{equation*}
Apply Taylor's theorem,
\begin{equation}
    \begin{split}
         &W^\top  \nabla g(W)
         \\
         =&
         \frac 1 2 \sum_{i,j=1}^{2K}
         \Lambda_{K,ii}^{-1} D^{ij} g(W) (w_i'-w_i) (w_j'-w_j)
         +
         \frac 1 4 \sum_{i,j,k=1}^{2K}
         \Lambda_{K,ii}^{-1} D^{ijk} g(W) (w_i'-w_i) (w_j'-w_j)(w_k'-w_k)
         \\
         &+
         \frac{1}{12} \sum_{i,j,k,l=1}^{2K}
         \Lambda_{K,ii}^{-1} D^{ijkl} g(t^*(W'-W)+W) (w_i'-w_i) (w_j'-w_j)(w_k'-w_k) (w_l'-w_l)
         \\
         =&
         \frac 1 2 
         \mytr
         [
         (W'-W)
         (W'-W)^\top \Lambda_K^{-\top} \nabla^2 g(W)
     ]
         +
         \frac 1 4 \sum_{i,j,k=1}^{2K}
         \Lambda_{K,ii}^{-1} D^{ijk} g(W) (w_i'-w_i) (w_j'-w_j)(w_k'-w_k)
         \\
         &+
         \frac{1}{12} \sum_{i,j,k,l=1}^{2K}
         \Lambda_{K,ii}^{-1} D^{ijkl} g(t^*(W'-W)+W) (w_i'-w_i) (w_j'-w_j)(w_k'-w_k) (w_l'-w_l),
    \end{split}
\end{equation}
where $t^*\in [0,1]$.
Also by exchangeability,
\begin{equation*}
    \myE[(W'-W) (W'-W)^\top] 
    =
    2\myE[ W(W-W')^\top] 
    =
    2\myE[ W W^\top \Lambda_K^\top] 
    =2\BV \Lambda_K^\top.
\end{equation*}
It follows that
\begin{equation*}
    \myE [\nabla^\top \BV \nabla g(W)] 
    =
    \myE \mytr [\BV \nabla^2 g(W)] 
    =
    \frac 1 2 \myE \mytr [ \myE [(W'-W)(W'-W)^\top] \Lambda_K^{-\top} \nabla^2 g(W)] 
\end{equation*}
Thus,
\begin{equation*}
    \begin{split}
    S=&\myE [\nabla^\top \BV \nabla g(W) -W^\top \nabla g(W)]\\
    =
&
    \frac 1 2 \myE \mytr [ \myE [(W'-W)(W'-W)^\top] \Lambda_K^{-\top} \nabla^2 g(W)] 
         -\frac 1 2 
\myE
         \mytr
         [
         (W'-W)
         (W'-W)^\top \Lambda_K^{-\top} \nabla^2 g(W)
     ]
     \\
         &-
         \frac 1 4 \myE
         \sum_{i,j,k=1}^{2K}
         \Lambda_{K,ii}^{-1} D^{ijk} g(W) (w_i'-w_i) (w_j'-w_j)(w_k'-w_k)
         \\
         &-
         \frac{1}{12} \myE
         \sum_{i,j,k,l=1}^{2K}
         \Lambda_{K,ii}^{-1} D^{ijkl} g(t^*(W'-W)+W) (w_i'-w_i) (w_j'-w_j)(w_k'-w_k) (w_l'-w_l).
    \end{split}
\end{equation*}


\end{proof}


\end{document}
